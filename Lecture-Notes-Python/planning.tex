\chapter{Planning}

\begin{Definition}[State Transition System]
A deterministic \emph{state transition system} is a triple $\Sigma = \langle S, A, \gamma \rangle$ such that:
\begin{itemize}
\item $S$ is the set of \emph{states},
\item $A$ is the set of \emph{actions}, and
\item $\gamma: S \times A \rightarrow S$ is the state transition function. \eod
\end{itemize}
\end{Definition}

\noindent
Executing an action $a \in A$ in a state $s_1$ yields a new state $s_2 = \gamma(s_1)$.  The function
$\gamma$ is extended to a function $\gamma^*$ that takes lists of actions as its second argument.
This definition is given inductively.
\begin{enumerate}
\item $\gamma^*(s, []) := s$,
\item $\gamma^*(s, [a] + r) := \gamma^*\bigl(\gamma(s,a), r\bigr)$.
\end{enumerate}
In order to simplify the notation we do not distinguish between $\gamma^*$ and $\gamma$, i.e. we
write $\gamma(s, l)$ instead of $\gamma^*(s, l)$.
  
\begin{Definition}[Planning Problem]
A \emph{planning problem} is a triple $\mathcal{P} = \langle \Sigma, s_0, g \rangle$ such that:
\begin{enumerate}
\item $\Sigma = \langle S, A, \gamma \rangle$ is a state transition system,
\item $s_0 \in S$ is the \emph{initial state}, and
\item $g \in S$ is the \emph{goal} state.
\end{enumerate}
\end{Definition}

A search problem is more abstract than a planning problem:  In a search problem, states are
connected in a way described by a relation $R$ that tells us whether two states are connected.  We
can go from state $s_1$ to state $s_2$ iff $\pair(s_1, s_2) \in R$.  In a planning problem, we
additionally have actions that tell us exactly what causes the state $s_1$ to be transformed into
the state $s_2$.

There are various ways to represent planning problems.  The three most common ones are as follows:
\begin{enumerate}
\item The \emph{set-theoretic} representation uses sets of propositional variables to represent a
      state.
\item The \emph{classical} representation is based on first order logic.
\item The \emph{state-variable} representation represents states by specifying the values that
      certain variables take.
\end{enumerate}
In this lecture, we will only have time to discuss the classical representation of the planning
problem.  The other representations are discussed in \cite{ghallab:2004}.


\section{The Classical Representation of a Planning Problem}
A \emph{first-order language} $\mathcal{L}$ is triple
\\[0.2cm]
\hspace*{1.3cm}
$\mathcal{L} = \langle P, C, V \rangle$, \quad where
\begin{enumerate}
\item $P$ is the set of \emph{predicate symbols},
\item $C$ is the set of \emph{constant symbols}, and
\item $V$ is the set of \emph{variable symbols}.
\end{enumerate}

In the classical representation of a planning problem, a state is a set of \emph{ground atoms} of the
language $\mathcal{L}$.  Here, a \emph{ground atom} is an atomic formula that does not contain
variables.  An atomic formula $p$ is true is a state $s$ iff $p \in s$.  Goals will be represented
by sets of literals.  A goal $g$ is \emph{satisfied} in a state $s$ iff there is a variable
substitution $\sigma$ such that every positive literal of $g\sigma$ is in $s$, while no negative
literal of $g\sigma$ is in $s$.  Here, we have used the so called \emph{closed world assumption}:
If $s$ is a state, i.e.~a set of ground atoms, then every ground atom that is not an element of $s$
is assumed to be false.

%%% Local Variables:
%%% mode: latex
%%% TeX-master: "artificial-intelligence"
%%% End:
